%!TEX program = xelatex
\documentclass[10pt,a4paper,sans]{moderncv}

\usepackage{amsmath,amssymb,amsthm} 
\usepackage{xcolor}
\usepackage{fontspec}
\usepackage{footmisc}

\moderncvstyle{classic} %classic/banking/casual
\moderncvcolor{green}
\moderncvicons{awesome}

% Font setup
\setsansfont{Metric-Light}[Ligatures=TeX,ItalicFont=Metric-LightItalic,BoldFont=Metric-Medium,BoldItalicFont=Metric-MediumItalic]
\setmonofont[BoldFont=GTPressuraMono-Bold,ItalicFont=GTPressuraMono-LightItalic]{GTPressuraMono-Light}

% adjust the page margins
\usepackage[scale=0.8]{geometry}

% personal data
\name{Ken}{Arroyo Ohori}
\title{Curriculum vit\ae{}}         
\address{Hacienda de Acambay 2}{Prado Coapa 3a Sección 14357}{Tlalpan, Mexico City, Mexico}
\email{k.ohori@tudelft.nl}
\homepage{3d.bk.tudelft.nl/ken}

\begin{document}

\recipient{Selection committee}{Faculty of Architecture and the Built Environment\\Julianalaan 134\\2628 BL Delft}
\date{November 13, 2020}
\opening{Dear selection committee,}
\closing{Best regards,}
\enclosure[Following]{curriculum vit\ae{}}          % use an optional argument to use a string other than "Enclosure", or redefine \enclname
\makelettertitle
I would hereby like to apply for the position \emph{Teacher Digitalisation and Design --- Theory, Methods and Concepts}, which was posted at the vacancies website of the Delft University of Technology. Because of my joint background in computer science and geomatics with years of research and teaching experience, I am certain that I am a good candidate for the position. In addition, I have long been interested in architecture and urbanism, working with friends in related projects, including one that won an honorary mention at a contest of the Venice Biennale in 2008\footnote{\url{http://www.newitalianblood.com/biennale2008/projects/287-1.html}}.

Since 2011, I have been working at TU Delft as a researcher in the 3D geoinformation group within the Department of Urbanism, first as a PhD candidate and then as a postdoc.
Because of this, as well as my MSc in Geomatics, I am familiar with all of the main theories, methods and concepts in geomatics, as well as with a variety of novel geomatics-related topics that are relevant for the position, including higher-dimensional modelling, GeoBIM integration and spatial data visualisation.
Also, because of my BSc in Computer science and technology, I am acquainted with digitalisation in general, as well as with the computational techniques that enable it, such as artificial intelligence.

Regarding my experience in education, I have been involved in teaching three courses within the MSc in Geomatics: GIS and cartography, Digital terrain modeling, and 3D modelling of the built environment.
I developed the latter one from scratch as a blended learning course\footnote{\url{https://3d.bk.tudelft.nl/courses/backup/geo1004/2020/}}, including building a website for it, writing course materials, editing videos, and providing both online and in-person help to students.
So far, I have finished one UTQ module (Develop), recently submitted the proofs of competence for two more (Supervise and Assess), and plan to take the final module in early 2021.

Regarding software, I have experience with many common GIS software packages, both open (including QGIS) and closed (including FME and ArcGIS).
As a complement, I also have experience from the developer side, including a Minor in Software engineering.
For the past 10 years, I have been using GIS libraries to build custom software to solve practical geospatial problems in different domains.
In order to do this, I have used both Python (12 years of experience) and C++ (17 years), as well as other programming languages.

I am passionate about teaching, which I often consider the best part of my job, so I would love to have the opportunity to develop and teach new courses.
Some landscapees and urbanists have followed courses that I have taught and enjoyed them, and I would welcome the opportunity to develop new courses specifically targetted for them.

\makeletterclosing

\clearpage
\maketitle

\section{Research interests}

\cvline{}{My research covers various aspects of geographic information, with a focus on the use of novel techniques to process data geometrically and topologically. These include: higher-dimensional (4D and higher) data structures and algorithms, GeoBIM integration, the validation and repair of geographic data, and visualising spatial data. I like theoretical work, but what I enjoy most is implementing my ideas into working prototypes and, when they are serious enough, into free software.}

\section{Work}
\cventry{2016--now}{Postdoc researcher}{Delft University of Technology}{the Netherlands}{}{}
\cventry{2011--2015}{PhD researcher}{Delft University of Technology}{the Netherlands}{}{}

\section{Education}
\cventry{2011--2016}{PhD}{Delft University of Technology}{the Netherlands}{}
{
  Title: \emph{Higher-dimensional modelling of geographic information}
}
\cventry{2008--2010}{MSc in Geomatics}{Delft University of Technology}{the Netherlands}{}
{
	Graduated with distinction \\
	Thesis: \emph{Validation and automatic repair of planar partitions using a constrained triangulation}
}
\cventry{2003--2007}{BSc in Computer Science and Technology}{Monterrey Institute of Technology and Higher Education, Mexico City Campus}{Mexico}{}
{
  Minor in Software Engineering
}

\section{Peer-reviewed publications}

\cvline{}{\textbf{Note}: PDFs are available at \url{https://3d.bk.tudelft.nl/ken/en/papers/}}

\cvline{2020}{\textbf{CityJSON in QGIS\@: development of an open-source plugin}. Stelios Vitalis, Ken Arroyo Ohori and Jantien Stoter. \emph{Transactions in GIS} 24(5), October 2020, pp. 1147--1164.}
\cvline{}{\textbf{azul\@: a fast and efficient 3D city model viewer for macOS}. Ken Arroyo Ohori. \emph{Transactions in GIS} 24(5), October 2020, pp. 1165--1184.}
\cvline{}{\textbf{GeoBIM for digital building permit process\@: learning from a case study in Rotterdam}. Francesca Noardo, Teng Wu, Ken Arroyo Ohori, Thomas Krijnen, Hasim Tezerdi and Jantien Stoter. \emph{15th 3D GeoInfo Conference}, ISPRS Annals of the Photogrammetry, Remote Sensing and Spatial Information Sciences, ISPRS, London, United Kingdom, September 2020.}
\cvline{}{\textbf{CityJSON + web = ninja}. Stelios Vitalis, Anna Labetski, Freek Boersma, Felix Dahle, Xiaoai Li, Ken Arroyo Ohori, Hugo Ledoux and Jantien Stoter. \emph{15th 3D GeoInfo Conference}, ISPRS Annals of the Photogrammetry, Remote Sensing and Spatial Information Sciences, ISPRS, London, United Kingdom, September 2020.}
\cvline{}{\textbf{Automatic conversion of CityGML to IFC}. Nebras Salheb, Ken Arroyo Ohori and Jantien Stoter. \emph{15th 3D GeoInfo Conference}, International Archives of the Photogrammetry, Remote Sensing and Spatial Information Sciences, ISPRS, London, United Kingdom, September 2020.}
\cvline{}{\textbf{The ISPRS-EuroSDR GeoBIM benchmark 2019}. Francesca Noardo, Ken Arroyo Ohori, Filip Biljecki, Claire Ellul, Lars Harrie, Thomas Krijnen, Margarita Kokla and Jantien Stoter. \emph{XXIV ISPRS Congress}, Commission V and Youth Forum, International Archives of the Photogrammetry, Remote Sensing and Spatial Information Sciences XLIII(B5), ISPRS, August 2020.}
\cvline{}{\textbf{Tools for BIM-GIS integration (IFC georeferencing and conversions): results from the GeoBIM Benchmark 2019}. Francesca Noardo, Lars Harrie, Ken Arroyo Ohori, Filip Biljecki, Claire Ellul, Thomas Krijnen, Helen Eriksson, Dogus Guler, Dean Hintz, Mojgan A. Jadidi, Maria Pla, Santi Sanchez, Ville-Pekka Soini, Rudi Stouffs, Jernej Tekavec and Jantien Stoter. \emph{ISPRS International Journal of Geo-Information} 9(9), August 2020.}
\cvline{}{\textbf{Validation and inference of geometrical relationships in IFC}. Thomas Krijnen, Francesca Noardo, Ken Arroyo Ohori, Hugo Ledoux and Jantien Stoter. \emph{Proceedings of the 37th International Conference of CIB W78}, Sao Paulo, Brazil, August 2020, pp. 98--111.}
\cvline{}{\textbf{Opportunities and challenges for GeoBIM in Europe\@: developing a building permits use-case to raise awareness and examine technical interoperability challenges}. Francesca Noardo, Claire Ellul, Lars Harrie, Ivar Overland, Masoome Shariat, Ken Arroyo Ohori and Jantien Stoter. \emph{Journal of Spatial Science} 65(2), May 2020, pp. 209--233.}
\cvline{2019}{\textbf{A data structure to incorporate versioning in 3D city models}. Stelios Vitalis, Anna Labetski, Ken Arroyo Ohori, Hugo Ledoux and Jantien Stoter. In R. Stouffs, F. Biljecki, K. H. Soon and V. Khoo (eds.), \emph{14th 3D GeoInfo Conference}, ISPRS Annals of the Photogrammetry, Remote Sensing and Spatial Information Sciences IV-4(W8), ISPRS, Singapore, September 2019, pp. 123--130.}
\cvline{}{\textbf{EuroSDR GeoBIM project\@: a study in Europe on how to use the potentials of BIM and Geo data in practice}. Francesca Noardo, Claire Ellul, Lars Harrie, Emmanuel Devys, Ken Arroyo Ohori, Perola Olsson and Jantien Stoter. In R. Stouffs, F. Biljecki, K. H. Soon and V. Khoo (eds.), \emph{14th 3D GeoInfo Conference}, International Archives of the Photogrammetry, Remote Sensing and Spatial Information Sciences XLII-4(W15), ISPRS, Singapore, September 2019, pp. 53--60.}
\cvline{}{\textbf{GeoBIM benchmark 2019\@: intermediate results}. Francesca Noardo, Filip Biljecki, Giorgio Agugiaro, Ken Arroyo Ohori, Claire Ellul, Lars Harrie and Jantien Stoter. In R. Stouffs, F. Biljecki, K. H. Soon and V. Khoo (eds.), \emph{14th 3D GeoInfo Conference}, International Archives of the Photogrammetry, Remote Sensing and Spatial Information Sciences XLII-4(W15), ISPRS, Singapore, September 2019, pp. 47--52.}
\cvline{}{\textbf{Incorporating topological representation in 3D city models}. Stelios Vitalis, Ken Arroyo Ohori and Jantien Stoter. \emph{ISPRS International Journal of Geo-Information 8(8)}, August 2019.}
\cvline{}{\textbf{The LandInfra standard and its role in solving the BIM-GIS quagmire}. Kavisha Kumar, Anna Labetski, Ken Arroyo Ohori, Hugo Ledoux and Jantien Stoter. \emph{Open Geospatial Data, Software and Standards} 4(5), July 2019.}
\cvline{}{\textbf{CityJSON\@: a compact and easy-to-use encoding of the CityGML data model}. Hugo Ledoux, Ken Arroyo Ohori, Kavisha Kumar, Balázs Dukai, Anna Labetski and Stelios Vitalis. \emph{Open Geospatial Data, Software and Standards} 4(4), June 2019.}
\cvline{}{\textbf{GeoBIM benchmark 2019: design and initial results}. Francesca Noardo, Ken Arroyo Ohori, Filip Biljecki, Thomas Krijnen, Claire Ellul, Lars Harrie and Jantien Stoter. In G. Vosselman, S. J. Oude Elberink and M. Y. Yang (eds.), \emph{ISPRS Geospatial Week 2019}, International Archives of the Photogrammetry, Remote Sensing and Spatial Information Sciences XLII-2(W13), ISPRS, June 2019, pp. 1339--1346.}
\cvline{}{\textbf{Harmonising the OGC standards for the built environment\@: a CityGML extension for LandInfra}. Kavisha Kumar, Anna Labetski, Ken Arroyo Ohori, Hugo Ledoux and Jantien Stoter. \emph{ISPRS International Journal of Geo-Information} 8(6), May 2019.}
\cvline{2018}{\textbf{A framework for the representation of two versions of a 3D model in 4D space}. Stelios Vitalis, Ken Arroyo Ohori and Jantien Stoter. In Ken Arroyo Ohori, Anna Labetski, Giorgio Agugiaro, Mila Koeva and Jantien Stoter (eds.), \emph{13th 3D Geoinfo Conference}, ISPRS Annals of the Photogrammetry, Remote Sensing and Spatial Information Sciences IV-4(W6), ISPRS, Delft, The Netherlands, October 2018, pp. 81--88.}
\cvline{}{\textbf{Modeling cities and landscapes in 3D with CityGML}. Ken Arroyo Ohori, Filip Biljecki, Kavisha Kumar, Hugo Ledoux and Jantien Stoter. In André Borrmann, Markus König, Christian Koch and Jakob Beetz (eds.), \emph{Building Information Modeling: Technology Foundations and Industry Practice}, Springer, September 2018, pp. 199--215.}
\cvline{}{\textbf{Processing BIM and GIS models in practice: experiences and recommendations from a GeoBIM project in the Netherlands}. Ken Arroyo Ohori, Abdoulaye Diakité, Thomas Krijnen, Hugo Ledoux and Jantien Stoter. \emph{ISPRS International Journal of Geo-Information} 7(8), August 2018.}
\cvline{}{\textbf{Topological reconstruction of 3D city models with preservation of semantics}. Stelios Vitalis, Ken Arroyo Ohori and Jantien Stoter. In A. Mansourian, P. Pilesjö, L. Harrie and R. von Lammeren (eds.), \emph{Geospatial Technologies for All: short papers, posters and poster abstracts of the 21th AGILE Conference on Geographic Information Science}. Lund University 12--15 June 2018, Lund, Sweden, June 2018.}
\cvline{}{\textbf{Essential means for urban computing: specification of web-based computing platforms for urban planning, a hitchhiker’s guide}. Pirouz Nourian, Carlos Martinez-Ortiz and Ken Arroyo Ohori. \emph{Urban Planning} 3(1), March 2018, pp. 47--57.}
\cvline{2017}{\textbf{Towards an integration of GIS and BIM data: what are the geometric and topological issues?}. Ken Arroyo Ohori, Filip Biljecki, Abdoulaye Diakité, Thomas Krijnen, Hugo Ledoux and Jantien Stoter. In M. Kalantari and A. Rajabifard (eds.), \emph{12th 3D Geoinfo Conference}, ISPRS Annals of the Photogrammetry, Remote Sensing and Spatial Information Sciences IV--4/W5, ISPRS, Melbourne, Australia, October 2017, pp. 1--8}
\cvline{}{\textbf{Visualising higher-dimensional space-time and space-scale objects as projections to R3}. Ken Arroyo Ohori, Hugo Ledoux and Jantien Stoter. \emph{PeerJ Computer Science} 3:e123, July 2017}
\cvline{}{\textbf{Modelling and manipulating spacetime objects in a true 4D model}. Ken Arroyo Ohori, Hugo Ledoux and Jantien Stoter. \emph{Journal of Spatial Information Science} 14, June 2017, pp. 61--93.}
\cvline{}{\textbf{Solving the horizontal conflation problem with a constrained Delaunay triangulation}. Hugo Ledoux and Ken Arroyo Ohori. \emph{Journal of Geographical Systems}, 19(1), January 2017, pp. 21--42.}
\cvline{2016}{\textbf{Defining simple nD operations based on prismatic nD objects}. Ken Arroyo Ohori, Hugo Ledoux and Jantien Stoter. In E. Dimopoulou and P. van Oosterom (eds.), \emph{11th 3D Geoinfo Conference}, ISPRS Annals of the Photogrammetry, Remote Sensing and Spatial Information Sciences IV--2/W1, ISPRS, Athens, Greece, October 2016, pp. 155--162.}
\cvline{}{\textbf{Population estimation using a 3D city model: a multi-scale country-wide study in the Netherlands}. Filip Biljecki, Ken Arroyo Ohori, Hugo Ledoux, Ravi Peters and Jantien Stoter. \emph{PLOS ONE} 11(6), June 2016.}
\cvline{}{\textbf{Voxelization algorithms for geospatial applications}. Pirouz Nourian, Romulo Gon\c{c}alves, Sisi Zlatanova, Ken Arroyo Ohori and Anh Vu Vo. \emph{MethodsX} 3, January 2016.}
\cvline{2015}{\textbf{Automatically enhancing CityGML LOD2 models with a corresponding indoor geometry}. Roeland Boeters, Ken Arroyo Ohori, Filip Biljecki and Sisi Zlatanova. \emph{International Journal of Geographical Information Science} 29(12), December 2015, pp. 2248--2268.}
\cvline{}{\textbf{Automatic semantic-preserving conversion between OBJ and CityGML}. Filip Biljecki and Ken Arroyo Ohori. In F. Biljecki and V. Tourre (eds.), \emph{Eurographics Workshop on Urban Data Modelling and Visualisation}, Eurographics Association, Delft, The Netherlands, November 2015, pp. 25--30.}
\cvline{}{\textbf{Storing a 3D city model, its levels of detail and the correspondences between objects as a 4D combinatorial map}. Ken Arroyo Ohori, Hugo Ledoux and Jantien Stoter. In Alias Abdul Rahman, Umit Isikdag and Francesc Ant\'on Castro (eds.), Joint International Geoinformation Conference 2015, 28--30 October 2015, Kuala Lumpur, Malaysia, ISPRS Annals of the Photogrammetry, Remote Sensing and Spatial Information Sciences II--2/W2, ISPRS, Kuala Lumpur, Malaysia, October 2015, pp. 1--8.}
\cvline{}{\textbf{Modelling a 3D city model and its levels of detail as a true 4D model}. Ken Arroyo Ohori, Hugo Ledoux, Filip Biljecki and Jantien Stoter. \emph{ISPRS International Journal of Geo-Information}, 4(3), September 2015, pp. 1055--1075.}
\cvline{}{\textbf{A dimension-independent extrusion algorithm using generalised maps}. Ken Arroyo Ohori, Hugo Ledoux and Jantien Stoter. \emph{International Journal of Geographical Information Science} 29(7), July 2015, pp. 1166--1186.}
\cvline{}{\textbf{An evaluation and classification of \emph{n}D topological data structures for the representation of objects in a higher-dimensional GIS}.\@ Ken Arroyo Ohori, Hugo Ledoux and Jantien Stoter. \emph{International Journal of Geographical Information Science} 29(5), May 2015, pp. 825--849.}
\cvline{2014}{\textbf{A triangulation-based approach to automatically repair GIS polygons}. Hugo Ledoux, Ken Arroyo Ohori and Martijn Meijers. \emph{Computers \& Geosciences} 66, May 2014, pp. 121--131.}
\cvline{}{\textbf{Constructing an \emph{n}-dimensional cell complex from a soup of \emph{(n-1)}-dimensional faces}. Ken Arroyo Ohori, Guillaume Damiand and Hugo Ledoux. In Prosenjit Gupta and Christos Zaroliagis (eds.), \emph{Applied Algorithms. First International Conference, ICAA 2014, Kolkata, India, January 13--15, 2014. Proceedings}, Lecture Notes in Computer Science 8321, Springer International Publishing Switzerland, Kolkata, India, January 2014, pp. 37--48.}
\cvline{2013}{\textbf{Using extrusion to generate higher-dimensional GIS datasets}. Ken Arroyo Ohori and Hugo Ledoux. In Craig Knoblock, Peer Kr\"oger, John Krumm, Markus Schneider and Peter Widmayer (eds.), \emph{SIGSPATIAL'13: Proceedings of the 21st ACM SIGSPATIAL International Conference on Advances in Geographic Information Systems}, ACM, Orlando, Florida, United States, November 2013, pp. 398--401.}
\cvline{}{\textbf{Modelling higher dimensional data for GIS using generalised maps}. Ken Arroyo Ohori, Hugo Ledoux and Jantien Stoter. In B. Murgante, S. Misra, M. Carlini, C. Torre, H. Q. Nguyen, D. Taniar, B. Apduhan and O. Gervasi (eds.), \emph{Computational Science and Its Applications --- ICCSA 2013. 13th International Conference, Ho Chi Minh City, Vietnam, June 24--27, 2013, Proceedings, Part I}, Lecture Notes in Computer Science 7971, Springer Berlin Heidelberg, June 2013, pp. 526--539.}
\cvline{}{\textbf{Representing the dual of objects in a four-dimensional GIS}.\@ Ken Arroyo Ohori, Pawel Boguslawski and Hugo Ledoux. In A. Abdul Rahman, P. Boguslawski, C. Gold and M. N. Said (eds.), \emph{Developments in Multidimensional Spatial Data Models}, Lecture Notes in Geoinformation and Cartography, Springer Berlin Heidelberg, Johor Bahru, Malaysia, May 2013, pp. 17--31.}
\cvline{}{\textbf{Manipulating higher dimensional spatial information}. Ken Arroyo Ohori, Filip Biljecki, Jantien Stoter and Hugo Ledoux. In Danny Vandenbroucke, B\'en\'edicte Bucher and Joep Crompvoets (eds.), \emph{Geographic Information Science at the Heart of Europe. Proceedings of the 16th AGILE International Conference on Geographic Information Science}, Leuven, Belgium, May 2013.}
\cvline{2012}{\textbf{Validation and automatic repair of planar partitions using a constrained triangulation}. Ken Arroyo Ohori, Hugo Ledoux and Martijn Meijers. \emph{Photogrammetrie, Fernerkundung, Geoinformation} 5, October 2012, pp. 613--630.}
\cvline{}{\textbf{Automatically repairing polygons and planar partitions with \emph{prepair} and \emph{pprepair}}. Ken Arroyo Ohori, Hugo Ledoux and Martijn Meijers. \emph{Proceedings of the 4th Open Source GIS UK Conference}, Nottingham, United Kingdom, September 2012.}
\cvline{}{\textbf{Integrating scale and space in 3D city models}. Jantien Stoter, Hugo Ledoux, Martijn Meijers and Ken Arroyo Ohori. In Jacynthe Pouliot, Sylvie Daniel, Fr\'ed\'eric Hubert and Alborz Zamyadi (eds.), \emph{Proceedings of the 7th International 3D GeoInfo Conference, International Archives of the Photogrammetry, Remote Sensing and Spatial Information Sciences} XXXVIII--4/C26, ISPRS, Québec City, Canada, May 2012, pp. 7--10.}
\cvline{}{\textbf{Automatically repairing invalid polygons with a constrained triangulation}. Hugo Ledoux, Ken Arroyo Ohori and Martijn Meijers. In J\'er\^ome Gensel, Didier Josselin and Danny Vandenbroucke (eds.), \emph{Multidisciplinary Research on Geographical Information in Europe and Beyond. Proceedings of the 15th AGILE International Conference on Geographic Information Science}, Avignon, France, April 2012, pp. 13--18.}
\cvline{2011}{\textbf{Edge-matching polygons with a constrained triangulation}. Hugo Ledoux and Ken Arroyo Ohori. \emph{Proceedings of GIS Ostrava 2011}, Ostrava, Czech Republic, January 2011, pp. 377--390.}

\section{Supervised thesis}

\cvline{2020}{\textbf{Knowledge sharing on Q\&A fora\@: challenges of automated interaction analysis}. Pablo Ruben. Master's thesis, Delft University of Technology, September 2020.}
\cvline{}{\textbf{Indoor positioning using augmented reality}. Laurens Oostwegel. Master's thesis, Delft University of Technology, July 2020.}
\cvline{}{\textbf{Automatic change detection in digital maps using aerial images and point clouds}. Felix Dahle. Master's thesis, Delft University of Technology, June 2020.}
\cvline{2019}{\textbf{Automatic conversion of CityGML to IFC}. Nebras Salheb. Master's thesis, Delft University of Technology, October 2019.}
\cvline{}{\textbf{Improving location accuracy of a crowdsourced weather station by using a point cloud\@: use case base Netatmo on the Hague}. Yixin Xu. Master's thesis, Delft University of Technology, June 2019.}
\cvline{2018}{\textbf{Simplification \& visualization of BIM models through Hololens}. Panagiotis Karydakis. Master's thesis, Delft University of Technology, October 2018.}
\cvline{2013}{\textbf{Automatic enhancement of CityGML LoD2 models with interiors and its usability for net internal area determination}. Roeland Boeters. Master's thesis, Delft University of Technology, June 2013.}

\section{Courses taught}
\cvline{2020}{\textbf{GEO1004 3D Modelling of the Built Environment} (2019--2020 Q3)}
\cvline{2019--2020}{\textbf{GEO1015 Digital Terrain Modelling} (2019--2020 Q2)}
\cvline{2018--2019}{\textbf{GEO1015 Digital Terrain Modelling} (2018--2019 Q2)}
\cvline{2014}{\textbf{GEO1002 Geographical Information Systems and Cartography} (2014--2015 Q1)}
\cvline{2013}{\textbf{GEO1002 Geographical Information Systems and Cartography} (2013--2014 Q1)}
\cvline{2011}{\textbf{Kickstarting your PhD}}

% \section{Guest lectures, supervised laboratories and other teaching tasks}

% \cvline{2015}{Lab work. \textbf{GEO1002 Geographical Information Systems and Cartography} (2015--2016 Q1).}
% \cvline{}{Lecture on volumetric representations of polyhedra. \textbf{GEO1004 3D Modelling of the Built Environment} (2014--2015 Q3).}
% \cvline{2013}{Course design and lab work. \textbf{GEO3001 Python Programming for Geomatics} (2013--2014 Q1).}
% \cvline{2012}{Lecture on validation and repair of 3D objects. \textbf{GEO1004 3D Modelling of the Built Environment} (2012--2013 Q2).}
% \cvline{}{Lab work and grading. \textbf{GEO1011 Introduction Geographical Information Systems} (2012--2013 Q1).}
% \cvline{}{Lab work and grading. \textbf{GEO1002 Geographical Information Systems and Cartography} (2012--2013 Q1).}
% \cvline{}{One lecture and lab work. \textbf{GEO1010 Python Programming for Geomatics} (2012--2013 Q1).}
% \cvline{2011}{Two lectures, lab work and grading. \textbf{GM1041 Introduction to GIS} (2011--2012 Q1).}
% \cvline{}{Course design and a half-day lecture. \textbf{Kickstarting your PhD}. December 2011.}

% \section{Conferences, workshops, competitions and courses}

% \cvline{2015}{Paper accepted: \textbf{3rd Eurographics Workshop on Urban Data Modelling and Visualisation}. Delft, the Netherlands. November 2015.}
% \cvline{}{Presented paper: \textbf{ISPRS WG II/2 Workshop}. Kuala Lumpur, Malaysia. October 2015.}
% \cvline{2014}{Presentation: \textbf{5D Workshop, GeoBuzz}. 's-Hertogenbosch, The Netherlands. November 2014.}
% % \cvline{}{Presentation: Geoinformation Technology and Governance, Capita Selecta lecture ABE010: Discipline-Related Skills for ABE.\@ Delft, The Netherlands. November 2014.}
% \cvline{}{Presented poster: \textbf{Lorentz Centre Workshop Geometric Algorithms in the Field}. Leiden, Netherlands. June 2014.}
% \cvline{}{Presented paper: \textbf{1st International Conference on Applied Algorithms}. Kolkata, India. January 2014.}
% \cvline{2013}{Presented paper: \textbf{21st ACM SIGSPATIAL International Conference on Advances in Geographic Information Systems}. Orlando, United States. November 2013.}
% \cvline{}{Presentation: \textbf{9th Dutch Computational Geometry Day}. Eindhoven University of Technology. Eindhoven, The Netherlands. October 2013.}
% \cvline{}{Presented paper: \textbf{13th International Conference on Computational Science and Its Applications}. Ho Chi Minh City, Vietnam. June 2013.}
% \cvline{}{Presented paper: \textbf{16th AGILE International Conference on Geographic Information Science}. Leuven, Belgium. May 2013.}
% \cvline{2012}{Presented paper: \textbf{GIN Symposium 2012}. Apeldoorn, The Netherlands. November 2012.}
% \cvline{}{Presented paper: \textbf{4th Open Source UK Conference}. Nottingham, United Kingdom. September 2012. \textbf{Best paper/presentation award}.}
% \cvline{}{Paper accepted: \textbf{7th International 3D GeoInfo Conference}. Quebec City, Canada. May 2012.}
% \cvline{}{Paper accepted: \textbf{Geospatial World Forum 2012}. Amsterdam, The Netherlands. April 2012.}
% \cvline{}{Paper accepted: \textbf{15th AGILE International Conference on Geographic Information Science}. Avignon, France. April 2012.}
% \cvline{}{Presentation: \textbf{8th Dutch Computational Geometry Day}. Utrecht University. Utrecht, The Netherlands. January 2012.}
% \cvline{2011}{Presented poster: \textbf{Summer School on High-Dimensional Geometric Computing 2011}. Aarhus, Denmark. August 2011.}
% \cvline{}{Paper accepted: \textbf{GIS Ostrava}. Ostrava, Czech Republic. January 2011.}
% \cvline{2010}{Presented poster: \textbf{International Geodetic Students Meeting 2010}. Zagreb, Croatia. May 2010.}
% \cvline{2009}{Course: \textbf{Natural language, engineering and the Internet}. Polytechnical University of Madrid, Madrid, Spain. November 2009.}
% \cvline{}{Attended: \textbf{International Geodetic Students Meeting 2009}, Zurich, Switzerland, April 2009.}
% \cvline{}{Attended course: \textbf{Text searching algorithms}. Czech Technical University. Prague, Czech Republic. March 2009.}
% \cvline{2008}{Entered competition: \textbf{Everyville. 11. Mostra Internazionale de Architettura, La Biennale di Venezia}. Venice, Italy. November 2008. \textbf{Honorary mention}.}
% \cvline{2007}{Attended course: \textbf{Satellite technologies}. Copenhagen University College of Engineering. Copenhagen, Denmark. July 2007.}
% \cvline{}{Attended course: \textbf{Wireless communications}. Jönköping University. Jönköping, Sweden. June 2007.}

\section{Open source software}

\cvline{2016--now}{\textbf{azul}, a 3D city model viewer for macOS.}
\cvline{2015--now}{\textbf{imbiber}, create nicely formatted HTML from BibTeX files directly from Jekyll.}
% \cvline{2014--2016}{\textbf{lcc-tools}, tools to construct and manipulate higher-dimensional linear cell complexes.}
\cvline{2010--now}{\textbf{pprepair}, (planar partition repair) ensures that a set of polygons form a valid planar partition, made of valid polygons and having no gaps or overlaps.}
\cvline{2010--now}{\textbf{prepair}, (polygon repair) takes a possibly invalid polygon, gives it a consistent interpretation and returns a valid polygon according to the OGC Simple Features and ISO 19107 rules.}

\section{Editorial board in journals}
\cvline{}{\textbf{International Journal of 3D Information Modeling}}

\section{Scientific committee member in conferences}

\cvline{2020}{\textbf{5th International Conference on Smart Data and Smart Cities}. Nice, France.}
\cvline{}{\textbf{15th 3D Geoinfo Conference}. London, United Kingdom.}
\cvline{}{\textbf{XXIVth ISPRS Congress}. Nice, France. June 2020.}
\cvline{2019}{\textbf{4th International Conference on Smart Data and Smart Cities}. Kuala Lumpur, Malaysia. October 2019.}
\cvline{}{\textbf{14th 3D Geoinfo Conference}. Singapore.}
\cvline{}{\textbf{ISPRS Student Consortium Summer School on Geospatial technologies for natural environment management and monitoring}. Wroclaw, Poland.}
\cvline{}{\textbf{International Workshop on Collaborative Crowdsourced Cloud Mapping and Geospatial Big Data}. Enschede, the Netherlands.}
\cvline{}{\textbf{Indoor 3D 2019}. Enschede, the Netherlands.}
\cvline{}{\textbf{1st Eurasian BIM Forum\@: Embracing the BIM Culture}. Istanbul, Turkey.}
\cvline{2018}{\textbf{13th 3D Geoinfo Conference}. Delft, the Netherlands.}
\cvline{}{\textbf{3rd International Conference on Smart Data and Smart Cities}. Delft, the Netherlands.}
\cvline{}{\textbf{ISPRS Technical Committee IV Symposium 2018}. Delft, the Netherlands.}
\cvline{2017}{\textbf{12th 3D Geoinfo Conference}. Melbourne, Australia.}
\cvline{}{\textbf{2nd International Conference on Smart Data and Smart Cities}. Puebla, Mexico.}
\cvline{}{\textbf{4th International GeoAdvances Workshop on Multi-dimensional \& Multi-scale Spatial Data Modeling}. Karabük, Turkey.}
\cvline{}{\textbf{3rd International Workshop on Indoor 3D}. Wuhan, China.}
\cvline{}{\textbf{ISPRS Workshop on Modelling and Managing Cartographic Data}. Washington, United States.}
\cvline{}{\textbf{AGILE 2017 Pre-conference Workshop on Bridging Space, Time, and Semantics in GIScience}. Wageningen, the Netherlands.}
\cvline{}{\textbf{GRASF Conference 2017}. Dubai, United Arab Emirates.}
\cvline{2016}{\textbf{4th Eurographics Workshop on Urban Data Modelling and Visualisation}. Li\`ege, Belgium.}
\cvline{}{\textbf{11th 3D Geoinfo Conference}. Athens, Greece.}
\cvline{}{\textbf{2nd International Conference in 3D Indoor Modelling and Navigation}. Athens, Greece.}
\cvline{2015}{\textbf{WITCOM 2015, Conferences and Workshops in Telematics and Computing}. Mexico City, Mexico.}
\cvline{}{\textbf{3rd Eurographics Workshop on Urban Data Modelling and Visualisation}. Delft, the Netherlands.}
\cvline{2014}{\textbf{1st International Congress on Telematics, Computing and Communications}. Mexico City, Mexico.}

\section{Reviews in journals}

\cvline{}{\textbf{Smart Cities} (2020)}
\cvline{}{\textbf{Built Environment} (2020)}
\cvline{}{\textbf{Applied Sciences} (2019, 2020)}
\cvline{}{\textbf{Automation in Construction} (2019)}
\cvline{}{\textbf{Environmental Pollution} (2018)}
\cvline{}{\textbf{Sensors} (2018, 2019)}
\cvline{}{\textbf{Transactions on Spatial Algorithms and Systems} (2017)}
\cvline{}{\textbf{Remote Sensing} (2017)}
\cvline{}{\textbf{ISPRS International Journal of Geo-Information} (2016, 2017, 2018, 2019, 2020)}
\cvline{}{\textbf{Journal of Geographical Systems} (2016, 2018, 2019)}
\cvline{}{\textbf{International Journal of Geographical Information Science} (2015, 2016, 2017, 2018, 2019, 2020)}
\cvline{}{\textbf{International Journal of 3D Information Modeling} (2014, 2016, 2018)}
\cvline{}{\textbf{Transactions in GIS} (2013, 2014, 2015, 2016, 2017, 2018, 2019, 2020)}
\cvline{}{\textbf{Computers \& Geosciences} (2012, 2013, 2014, 2015, 2016, 2017, 2018, 2019)}

\section{Skills}

\subsection{Computer-related}
\cvline{Operating Systems}{Familiar with all recent versions of Windows, macOS and various Linux distributions. Others as a user, but with little programming experience.}
\cvline{Programming}{Familiar with C, C++, CSS, HTML, Objective-C, PHP, Python, Ruby and Swift. Others in a lesser degree (e.g.\ Java, Javascript, Lisp, Matlab, Scheme, Perl, Prolog). Use of other relevant software, libraries and frameworks for computational geometry (e.g.\ CGAL, openCASCADE), graphics and visualisation (e.g.\ OpenGL, Metal), debugging and unit testing, parsing, I/O in various file formats, and user interfaces, among others.}
\cvline{Software}{Various office suites, databases with spatial extensions, web servers, GIS, typesetting with (Xe)LaTeX, image/video editing packages and drawing programs, among others.}

\subsection{Languages}

\cvline{Spanish}{native speaker}
\cvline{English}{expert (IELTS 8.5 and paper based TOEFL 670)}
\cvline{Dutch}{intermediate}
\cvline{Japanese}{intermediate (3-kyu)}
% \cvline{French}{very basic}
% \cvline{German}{very basic}

\section{References}
\cvline{}{Available upon request}

\end{document}

%!TEX program = xelatex
\documentclass[11pt,a4paper,sans]{moderncv}

\usepackage{amsmath,amssymb,amsthm} 
\usepackage{xcolor}
\usepackage{fontspec}

\moderncvstyle{classic} %classic/banking/casual
\moderncvcolor{green}
\moderncvicons{awesome}

% Font setup
\setsansfont{Metric-Light}[Ligatures=TeX,ItalicFont=Metric-LightItalic,BoldFont=Metric-Medium,BoldItalicFont=Metric-MediumItalic]
\setmonofont[BoldFont=GTPressuraMono-Bold,ItalicFont=GTPressuraMono-LightItalic]{GTPressuraMono-Light}

% adjust the page margins
\usepackage[scale=0.8]{geometry}

% personal data
\name{Ken}{Arroyo Ohori}
\title{Curriculum vitae}               
\address{Delft University of Technology}{Julianalaan 134, Delft 2628 BL,the Netherlands}{}                  
\email{g.a.k.arroyoohori@tudelft.nl}
\homepage{tudelft.nl/kenohori}

\begin{document}
\maketitle

\section{Research interests}

\cvline{}{The main focus of my research is higher-dimensional (4D and higher) data models, data structures and algorithms for Geographic Information Systems, using these dimensions to model not only space, but also other characteristics such as time and scale. However, I also work on other topics that combine geometric or topological computing with spatial information, such as geometric modelling and processing, the validation and repair of geographic data, and visualising spatial data. I like theoretical work, but what I enjoy most is implementing my ideas into working prototypes and, when they are serious enough, into free software.}

\section{Research projects}

\cvline{}{From 2011 until 2017, I worked as a PhD candidate and postdoc in the NWO/STW project \emph{5D Data Modelling}, which aimed to integrate the multi-dimensional characteristics of geographic data, i.e. 2D/3D, time and scale at a fundamental level of data modelling.
My role in that project was realising the foundations of a higher-dimensional GIS based on this aim, working on developing data models, data structures and algorithms.
In 2017--2018, I worked as one of the main researchers in a Geo-BIM integration project, which involved two research groups on BIM and 3D GIS (in TU Eindhoven and TU Delft), the two respective national standardisation bodies (BIM Loket and Geonovum) and several users who have a high interest in closer BIM/GIS integration, i.e. Rijkswaterstaat, Kadaster and the cities of The Hague and Rotterdam.
The aim of the project was to develop an interface to process complex architectural IFC models in an automated fashion, such as performing automated tests on them and converting them to CityGML\@.}

\section{Conferences}

\cvline{}{My work has been presented at a variety of international GIS and computer science conferences, including 3D GeoInfo, ACM SIGSPATIAL, ICCSA, Agile, and ISPRS workshops.
I have also been in the scientific committee of several conferences, including Eurographics UDMV, Indoor 3D, 3D GeoInfo, GRASF, Agile, GeoAdvances, UDMS/SDSC, and some ISPRS workshops and symposia.
In addition, I was a local organiser for the 3D GeoInfo 2018 conference in Delft, the Netherlands.}

\section{Publications}

\cvline{}{I have authored 29 peer-reviewed papers, including 13 journal publications, which have been published in the main journals at the junction of computer science and GIS, including IJGIS, Computers \& Geosciences IJGI, JGSY, PLOS ONE, PeerJ Computer Science, and JOSIS\@.
All of them are available in my personal website.}

% \section{Work}
% \cventry{2016--now}{Postdoc researcher}{Delft University of Technology}{the Netherlands}{}{}
% \cventry{2011--2015}{PhD researcher}{Delft University of Technology}{the Netherlands}{}{}

% \section{Education}
% \cventry{2011--2016}{PhD}{Delft University of Technology}{the Netherlands}{}
% {
%   Title: \emph{Higher-dimensional modelling of geographic information}
% }
% \cventry{2008--2010}{MSc in Geomatics}{Delft University of Technology}{the Netherlands}{}
% {
% 	Graduated with distinction \\
% 	Thesis: \emph{Validation and automatic repair of planar partitions using a constrained triangulation}
% }
% \cventry{2003--2007}{BSc in Computer Science and Technology}{Monterrey Institute of Technology and Higher Education, Mexico City Campus}{Mexico}{}
% {
%   Minor in Software Engineering
% }

% \section{Peer-reviewed publications}

% \cvline{}{\textbf{Note}: PDFs are available at \url{https://3d.bk.tudelft.nl/ken/en/papers/}}

% \cvline{2018}{\textbf{A framework for the representation of two versions of a 3D model in 4D space}. Stelios Vitalis, Ken Arroyo Ohori and Jantien Stoter. In Ken Arroyo Ohori, Anna Labetski, Giorgio Agugiaro, Mila Koeva and Jantien Stoter (eds.), \emph{13th 3D Geoinfo Conference}, ISPRS Annals of the Photogrammetry, Remote Sensing and Spatial Information Sciences IV-4(W6), ISPRS, Delft, The Netherlands, October 2018, pp. 81--88.}
% \cvline{}{\textbf{Modeling cities and landscapes in 3D with CityGML}. Ken Arroyo Ohori, Filip Biljecki, Kavisha Kumar, Hugo Ledoux and Jantien Stoter. In André Borrmann, Markus König, Christian Koch and Jakob Beetz (eds.), \emph{Building Information Modeling: Technology Foundations and Industry Practice}, Springer, September 2018, pp. 199--215.}
% \cvline{}{\textbf{Processing BIM and GIS models in practice: experiences and recommendations from a GeoBIM project in the Netherlands}. Ken Arroyo Ohori, Abdoulaye Diakité, Thomas Krijnen, Hugo Ledoux and Jantien Stoter. \emph{ISPRS International Journal of Geo-Information} 7(8), August 2018.}
% \cvline{}{\textbf{Topological reconstruction of 3D city models with preservation of semantics}. Stelios Vitalis, Ken Arroyo Ohori and Jantien Stoter. In A. Mansourian, P. Pilesjö, L. Harrie and R. von Lammeren (eds.), \emph{Geospatial Technologies for All: short papers, posters and poster abstracts of the 21th AGILE Conference on Geographic Information Science}. Lund University 12--15 June 2018, Lund, Sweden, June 2018.}
% \cvline{}{\textbf{Essential means for urban computing: specification of web-based computing platforms for urban planning, a hitchhiker’s guide}. Pirouz Nourian, Carlos Martinez-Ortiz and Ken Arroyo Ohori. \emph{Urban Planning} 3(1), March 2018, pp. 47--57.}
% \cvline{2017}{\textbf{Towards an integration of GIS and BIM data: what are the geometric and topological issues?}. Ken Arroyo Ohori, Filip Biljecki, Abdoulaye Diakité, Thomas Krijnen, Hugo Ledoux and Jantien Stoter. In M. Kalantari and A. Rajabifard (eds.), \emph{12th 3D Geoinfo Conference}, ISPRS Annals of the Photogrammetry, Remote Sensing and Spatial Information Sciences IV--4/W5, ISPRS, Melbourne, Australia, October 2017, pp. 1--8}
% \cvline{}{\textbf{Visualising higher-dimensional space-time and space-scale objects as projections to R3}. Ken Arroyo Ohori, Hugo Ledoux and Jantien Stoter. \emph{PeerJ Computer Science} 3:e123, July 2017}
% \cvline{}{\textbf{Modelling and manipulating spacetime objects in a true 4D model}. Ken Arroyo Ohori, Hugo Ledoux and Jantien Stoter. \emph{Journal of Spatial Information Science} 14, June 2017, pp. 61--93.}
% \cvline{}{\textbf{Solving the horizontal conflation problem with a constrained Delaunay triangulation}. Hugo Ledoux and Ken Arroyo Ohori. \emph{Journal of Geographical Systems}, 19(1), January 2017, pp. 21--42.}
% \cvline{2016}{\textbf{Defining simple nD operations based on prismatic nD objects}. Ken Arroyo Ohori, Hugo Ledoux and Jantien Stoter. In E. Dimopoulou and P. van Oosterom (eds.), \emph{11th 3D Geoinfo Conference}, ISPRS Annals of the Photogrammetry, Remote Sensing and Spatial Information Sciences IV--2/W1, ISPRS, Athens, Greece, October 2016, pp. 155--162.}
% \cvline{}{\textbf{Population estimation using a 3D city model: a multi-scale country-wide study in the Netherlands}. Filip Biljecki, Ken Arroyo Ohori, Hugo Ledoux, Ravi Peters and Jantien Stoter. \emph{PLOS ONE} 11(6), June 2016.}
% \cvline{}{\textbf{Voxelization algorithms for geospatial applications}. Pirouz Nourian, Romulo Gon\c{c}alves, Sisi Zlatanova, Ken Arroyo Ohori and Anh Vu Vo. \emph{MethodsX} 3, January 2016.}
% \cvline{2015}{\textbf{Automatically enhancing CityGML LOD2 models with a corresponding indoor geometry}. Roeland Boeters, Ken Arroyo Ohori, Filip Biljecki and Sisi Zlatanova. \emph{International Journal of Geographical Information Science} 29(12), December 2015, pp. 2248–2268.}
% \cvline{}{\textbf{Automatic semantic-preserving conversion between OBJ and CityGML}. Filip Biljecki and Ken Arroyo Ohori. In F. Biljecki and V. Tourre (eds.), \emph{Eurographics Workshop on Urban Data Modelling and Visualisation}, Eurographics Association, Delft, The Netherlands, November 2015, pp. 25--30.}
% \cvline{}{\textbf{Storing a 3D city model, its levels of detail and the correspondences between objects as a 4D combinatorial map}. Ken Arroyo Ohori, Hugo Ledoux and Jantien Stoter. In Alias Abdul Rahman, Umit Isikdag and Francesc Ant\'on Castro (eds.), Joint International Geoinformation Conference 2015, 28--30 October 2015, Kuala Lumpur, Malaysia, ISPRS Annals of the Photogrammetry, Remote Sensing and Spatial Information Sciences II--2/W2, ISPRS, Kuala Lumpur, Malaysia, October 2015, pp. 1--8.}
% \cvline{}{\textbf{Modelling a 3D city model and its levels of detail as a true 4D model}. Ken Arroyo Ohori, Hugo Ledoux, Filip Biljecki and Jantien Stoter. \emph{ISPRS International Journal of Geo-Information}, 4(3), September 2015, pp. 1055--1075.}
% \cvline{}{\textbf{A dimension-independent extrusion algorithm using generalised maps}. Ken Arroyo Ohori, Hugo Ledoux and Jantien Stoter. \emph{International Journal of Geographical Information Science} 29(7), July 2015, pp. 1166--1186.}
% \cvline{}{\textbf{An evaluation and classification of \emph{n}D topological data structures for the representation of objects in a higher-dimensional GIS}.\@ Ken Arroyo Ohori, Hugo Ledoux and Jantien Stoter. \emph{International Journal of Geographical Information Science} 29(5), May 2015, pp. 825--849.}
% \cvline{2014}{\textbf{A triangulation-based approach to automatically repair GIS polygons}. Hugo Ledoux, Ken Arroyo Ohori and Martijn Meijers. \emph{Computers \& Geosciences} 66, May 2014, pp. 121--131.}
% \cvline{}{\textbf{Constructing an \emph{n}-dimensional cell complex from a soup of \emph{(n-1)}-dimensional faces}. Ken Arroyo Ohori, Guillaume Damiand and Hugo Ledoux. In Prosenjit Gupta and Christos Zaroliagis (eds.), \emph{Applied Algorithms. First International Conference, ICAA 2014, Kolkata, India, January 13--15, 2014. Proceedings}, Lecture Notes in Computer Science 8321, Springer International Publishing Switzerland, Kolkata, India, January 2014, pp. 37--48.}
% \cvline{2013}{\textbf{Using extrusion to generate higher-dimensional GIS datasets}. Ken Arroyo Ohori and Hugo Ledoux. In Craig Knoblock, Peer Kr\"oger, John Krumm, Markus Schneider and Peter Widmayer (eds.), \emph{SIGSPATIAL'13: Proceedings of the 21st ACM SIGSPATIAL International Conference on Advances in Geographic Information Systems}, ACM, Orlando, Florida, United States, November 2013, pp. 398--401.}
% \cvline{}{\textbf{Modelling higher dimensional data for GIS using generalised maps}. Ken Arroyo Ohori, Hugo Ledoux and Jantien Stoter. In B. Murgante, S. Misra, M. Carlini, C. Torre, H. Q. Nguyen, D. Taniar, B. Apduhan and O. Gervasi (eds.), \emph{Computational Science and Its Applications --- ICCSA 2013. 13th International Conference, Ho Chi Minh City, Vietnam, June 24–27, 2013, Proceedings, Part I}, Lecture Notes in Computer Science 7971, Springer Berlin Heidelberg, June 2013, pp. 526--539.}
% \cvline{}{\textbf{Representing the dual of objects in a four-dimensional GIS}.\@ Ken Arroyo Ohori, Pawel Boguslawski and Hugo Ledoux. In A. Abdul Rahman, P. Boguslawski, C. Gold and M. N. Said (eds.), \emph{Developments in Multidimensional Spatial Data Models}, Lecture Notes in Geoinformation and Cartography, Springer Berlin Heidelberg, Johor Bahru, Malaysia, May 2013, pp. 17--31.}
% \cvline{}{\textbf{Manipulating higher dimensional spatial information}. Ken Arroyo Ohori, Filip Biljecki, Jantien Stoter and Hugo Ledoux. In Danny Vandenbroucke, B\'en\'edicte Bucher and Joep Crompvoets (eds.), \emph{Geographic Information Science at the Heart of Europe. Proceedings of the 16th AGILE International Conference on Geographic Information Science}, Leuven, Belgium, May 2013.}
% \cvline{2012}{\textbf{Validation and automatic repair of planar partitions using a constrained triangulation}. Ken Arroyo Ohori, Hugo Ledoux and Martijn Meijers. \emph{Photogrammetrie, Fernerkundung, Geoinformation} 5, October 2012, pp. 613--630.}
% \cvline{}{\textbf{Automatically repairing polygons and planar partitions with \emph{prepair} and \emph{pprepair}}. Ken Arroyo Ohori, Hugo Ledoux and Martijn Meijers. \emph{Proceedings of the 4th Open Source GIS UK Conference}, Nottingham, United Kingdom, September 2012.}
% \cvline{}{\textbf{Integrating scale and space in 3D city models}. Jantien Stoter, Hugo Ledoux, Martijn Meijers and Ken Arroyo Ohori. In Jacynthe Pouliot, Sylvie Daniel, Fr\'ed\'eric Hubert and Alborz Zamyadi (eds.), \emph{Proceedings of the 7th International 3D GeoInfo Conference, International Archives of the Photogrammetry, Remote Sensing and Spatial Information Sciences} XXXVIII--4/C26, ISPRS, Québec City, Canada, May 2012, pp. 7--10.}
% \cvline{}{\textbf{Automatically repairing invalid polygons with a constrained triangulation}. Hugo Ledoux, Ken Arroyo Ohori and Martijn Meijers. In J\'er\^ome Gensel, Didier Josselin and Danny Vandenbroucke (eds.), \emph{Multidisciplinary Research on Geographical Information in Europe and Beyond. Proceedings of the 15th AGILE International Conference on Geographic Information Science}, Avignon, France, April 2012, pp. 13--18.}
% \cvline{2011}{\textbf{Edge-matching polygons with a constrained triangulation}. Hugo Ledoux and Ken Arroyo Ohori. \emph{Proceedings of GIS Ostrava 2011}, Ostrava, Czech Republic, January 2011, pp. 377--390.}

% \section{Supervised thesis}

% \cvline{2013}{\textbf{Automatic enhancement of CityGML LoD2 models with interiors and its usability for net internal area determination}. Roeland Boeters. Master's thesis, Delft University of Technology, June 2013.}

% \section{Full courses given}

% \cvline{2018--2019}{\textbf{GEO1015 Digital Terrain Modelling} (2018--2019 Q2)}
% \cvline{2014}{\textbf{GEO1002 Geographical Information Systems and Cartography} (2014--2015 Q1)}
% \cvline{2013}{\textbf{GEO1002 Geographical Information Systems and Cartography} (2013--2014 Q1)}

% % \section{Guest lectures, supervised laboratories and other teaching tasks}

% % \cvline{2015}{Lab work. \textbf{GEO1002 Geographical Information Systems and Cartography} (2015--2016 Q1).}
% % \cvline{}{Lecture on volumetric representations of polyhedra. \textbf{GEO1004 3D Modelling of the Built Environment} (2014--2015 Q3).}
% % \cvline{2013}{Course design and lab work. \textbf{GEO3001 Python Programming for Geomatics} (2013--2014 Q1).}
% % \cvline{2012}{Lecture on validation and repair of 3D objects. \textbf{GEO1004 3D Modelling of the Built Environment} (2012--2013 Q2).}
% % \cvline{}{Lab work and grading. \textbf{GEO1011 Introduction Geographical Information Systems} (2012--2013 Q1).}
% % \cvline{}{Lab work and grading. \textbf{GEO1002 Geographical Information Systems and Cartography} (2012--2013 Q1).}
% % \cvline{}{One lecture and lab work. \textbf{GEO1010 Python Programming for Geomatics} (2012--2013 Q1).}
% % \cvline{2011}{Two lectures, lab work and grading. \textbf{GM1041 Introduction to GIS} (2011--2012 Q1).}
% % \cvline{}{Course design and a half-day lecture. \textbf{Kickstarting your PhD}. December 2011.}

% % \section{Conferences, workshops, competitions and courses}

% % \cvline{2015}{Paper accepted: \textbf{3rd Eurographics Workshop on Urban Data Modelling and Visualisation}. Delft, the Netherlands. November 2015.}
% % \cvline{}{Presented paper: \textbf{ISPRS WG II/2 Workshop}. Kuala Lumpur, Malaysia. October 2015.}
% % \cvline{2014}{Presentation: \textbf{5D Workshop, GeoBuzz}. 's-Hertogenbosch, The Netherlands. November 2014.}
% % % \cvline{}{Presentation: Geoinformation Technology and Governance, Capita Selecta lecture ABE010: Discipline-Related Skills for ABE.\@ Delft, The Netherlands. November 2014.}
% % \cvline{}{Presented poster: \textbf{Lorentz Centre Workshop Geometric Algorithms in the Field}. Leiden, Netherlands. June 2014.}
% % \cvline{}{Presented paper: \textbf{1st International Conference on Applied Algorithms}. Kolkata, India. January 2014.}
% % \cvline{2013}{Presented paper: \textbf{21st ACM SIGSPATIAL International Conference on Advances in Geographic Information Systems}. Orlando, United States. November 2013.}
% % \cvline{}{Presentation: \textbf{9th Dutch Computational Geometry Day}. Eindhoven University of Technology. Eindhoven, The Netherlands. October 2013.}
% % \cvline{}{Presented paper: \textbf{13th International Conference on Computational Science and Its Applications}. Ho Chi Minh City, Vietnam. June 2013.}
% % \cvline{}{Presented paper: \textbf{16th AGILE International Conference on Geographic Information Science}. Leuven, Belgium. May 2013.}
% % \cvline{2012}{Presented paper: \textbf{GIN Symposium 2012}. Apeldoorn, The Netherlands. November 2012.}
% % \cvline{}{Presented paper: \textbf{4th Open Source UK Conference}. Nottingham, United Kingdom. September 2012. \textbf{Best paper/presentation award}.}
% % \cvline{}{Paper accepted: \textbf{7th International 3D GeoInfo Conference}. Quebec City, Canada. May 2012.}
% % \cvline{}{Paper accepted: \textbf{Geospatial World Forum 2012}. Amsterdam, The Netherlands. April 2012.}
% % \cvline{}{Paper accepted: \textbf{15th AGILE International Conference on Geographic Information Science}. Avignon, France. April 2012.}
% % \cvline{}{Presentation: \textbf{8th Dutch Computational Geometry Day}. Utrecht University. Utrecht, The Netherlands. January 2012.}
% % \cvline{2011}{Presented poster: \textbf{Summer School on High-Dimensional Geometric Computing 2011}. Aarhus, Denmark. August 2011.}
% % \cvline{}{Paper accepted: \textbf{GIS Ostrava}. Ostrava, Czech Republic. January 2011.}
% % \cvline{2010}{Presented poster: \textbf{International Geodetic Students Meeting 2010}. Zagreb, Croatia. May 2010.}
% % \cvline{2009}{Course: \textbf{Natural language, engineering and the Internet}. Polytechnical University of Madrid, Madrid, Spain. November 2009.}
% % \cvline{}{Attended: \textbf{International Geodetic Students Meeting 2009}, Zurich, Switzerland, April 2009.}
% % \cvline{}{Attended course: \textbf{Text searching algorithms}. Czech Technical University. Prague, Czech Republic. March 2009.}
% % \cvline{2008}{Entered competition: \textbf{Everyville. 11. Mostra Internazionale de Architettura, La Biennale di Venezia}. Venice, Italy. November 2008. \textbf{Honorary mention}.}
% % \cvline{2007}{Attended course: \textbf{Satellite technologies}. Copenhagen University College of Engineering. Copenhagen, Denmark. July 2007.}
% % \cvline{}{Attended course: \textbf{Wireless communications}. Jönköping University. Jönköping, Sweden. June 2007.}

% \section{Open source software}

% \cvline{2016--now}{\textbf{azul}, a CityGML viewer for macOS.}
% \cvline{2015--now}{\textbf{imbiber}, create nicely formatted HTML from BibTeX files directly from Jekyll.}
% \cvline{2014--2016}{\textbf{lcc-tools}, tools to construct and manipulate higher-dimensional linear cell complexes.}
% \cvline{2010--now}{\textbf{pprepair}, (planar partition repair) ensures that a set of polygons form a valid planar partition, made of valid polygons and having no gaps or overlaps.}
% \cvline{2010--now}{\textbf{prepair}, (polygon repair) takes a possibly invalid polygon, gives it a consistent interpretation and returns a valid polygon according to the OGC Simple Features and ISO 19107 rules.}

% \section{Editorial board in journals}
% \cvline{}{\textbf{International Journal of 3D Information Modeling}}

% \section{Scientific committee member in conferences}

% \cvline{2018}{\textbf{13th 3D Geoinfo Conference}. Delft, the Netherlands.}
% \cvline{}{\textbf{3rd International Conference on Smart Data and Smart Cities}. Delft, the Netherlands.}
% \cvline{}{\textbf{ISPRS Technical Committee IV Symposium 2018}. Delft, the Netherlands.}
% \cvline{2017}{\textbf{12th 3D Geoinfo Conference}. Melbourne, Australia.}
% \cvline{}{\textbf{2nd International Conference on Smart Data and Smart Cities}. Puebla, Mexico.}
% \cvline{}{\textbf{4th International GeoAdvances Workshop on Multi-dimensional \& Multi-scale Spatial Data Modeling}. Karabük, Turkey.}
% \cvline{}{\textbf{3rd International Workshop on Indoor 3D}. Wuhan, China.}
% \cvline{}{\textbf{ISPRS Workshop on Modelling and Managing Cartographic Data}. Washington, United States.}
% \cvline{}{\textbf{AGILE 2017 Pre-conference Workshop on Bridging Space, Time, and Semantics in GIScience}. Wageningen, the Netherlands.}
% \cvline{}{\textbf{GRASF Conference 2017}. Dubai, United Arab Emirates.}
% \cvline{2016}{\textbf{4th Eurographics Workshop on Urban Data Modelling and Visualisation}. Li\`ege, Belgium.}
% \cvline{}{\textbf{11th 3D Geoinfo Conference}. Athens, Greece.}
% \cvline{}{\textbf{2nd International Conference in 3D Indoor Modelling and Navigation}. Athens, Greece.}
% \cvline{2015}{\textbf{WITCOM 2015, Conferences and Workshops in Telematics and Computing}. Mexico City, Mexico.}
% \cvline{}{\textbf{3rd Eurographics Workshop on Urban Data Modelling and Visualisation}. Delft, the Netherlands.}
% \cvline{2014}{\textbf{1st International Congress on Telematics, Computing and Communications}. Mexico City, Mexico.}

% \section{Reviews in journals}

% \cvline{}{\textbf{Environmental Pollution} (2018)}
% \cvline{}{\textbf{Sensors} (2018)}
% \cvline{}{\textbf{Transactions on Spatial Algorithms and Systems} (2017)}
% \cvline{}{\textbf{Remote Sensing} (2017)}
% \cvline{}{\textbf{ISPRS International Journal of Geo-Information} (2016, 2017, 2018)}
% \cvline{}{\textbf{Journal of Geographical Systems} (2016, 2018)}
% \cvline{}{\textbf{International Journal of Geographical Information Science} (2015, 2016, 2017)}
% \cvline{}{\textbf{International Journal of 3D Information Modeling} (2014, 2016, 2018)}
% \cvline{}{\textbf{Transactions in GIS} (2013, 2014, 2015, 2016, 2017, 2018)}
% \cvline{}{\textbf{Computers \& Geosciences} (2012, 2013, 2014, 2015, 2016, 2017, 2018)}

% \section{Skills}

% \subsection{Computer-related}
% \cvline{Operating Systems}{Familiar with all recent versions of Windows, Mac OS and various Linux distributions. Others as a user, but with little programming experience.}
% \cvline{Programming}{Familiar with C, C++, CSS, HTML, Objective-C, PHP, Python and Ruby. Others in a lesser degree (e.g.\ Java, Javascript, Lisp, Matlab, Scheme, Perl, Prolog). Use of other relevant software, libraries and frameworks for computational geometry (e.g.\ CGAL, openCASCADE), graphics and visualisation (e.g.\ OpenGL), debugging and unit testing, parsing, I/O in various file formats, and user interfaces, among others.}
% \cvline{Software}{Various office suites (e.g.\ iWork, MS Office and OpenOffice.org), databases (e.g. MySQL, Oracle and PostgreSQL) with spatial extensions, web servers (Apache and nginx), typesetting with (Xe)LaTeX, image editing packages and drawing programs (e.g. Aperture, OmniGraffle and Adobe Photoshop), among others.}

% \subsection{Languages}

% \cvline{Spanish}{native speaker}
% \cvline{English}{expert (IELTS 8.5 and paper based TOEFL 670)}
% \cvline{Japanese}{intermediate (3-kyu)}
% \cvline{Dutch}{intermediate}
% % \cvline{French}{very basic}
% % \cvline{German}{very basic}

% \section{References}
% \cvline{}{Available upon request}

\end{document}
